%%%%%%%%%%%%%%%%%%%%%%%%%%%%%%%%%%%%%%
%%%%%%  eigene Packages  %%%%%%%%%%%%%
%%%%%%%%%%%%%%%%%%%%%%%%%%%%%%%%%%%%%%

\usepackage[ngerman,pdfauthor={Ole Kiefer},  pdfauthor={Ole Kiefer}, pdftitle={Collaborative Filtering}, pdfkeywords={Bachelor Arbeit, Collaborative Filtering, Kollaboratives Filtern, HHU} breaklinks=true,baseurl={http://www.bretschneidernet.de/tips/thesislatex.html}]{hyperref}





\usepackage{amsmath,amssymb,marvosym} % Mathesachen
\usepackage{listings}


% Bilder
\usepackage{placeins} % Grafik Platzierung
\usepackage{graphicx} % Bilder
\usepackage{color} % Farben
\graphicspath{{bilder/}}
\DeclareGraphicsExtensions{.pdf,.png,.jpg} % bevorzuge pdf-Dateien
\usepackage{subcaption}  % mehrere Abbildungen nebeneinander/übereinander

\usepackage[all]{hypcap} % Beim Klicken auf Links zum Bild und nicht zu Caption gehen


\newcommand{\todo}[1]{
	{\colorbox{red}{ TODO: #1 }}
}
\newcommand{\todotext}[1]{
	{\color{red} TODO: #1} \normalfont
}