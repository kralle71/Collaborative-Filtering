%%%%%%%%%%%%%%%%%%%%%%%%%%%%%%%%%%%%%%%%%%%%%%%%%%%%%%%%%%%%%%%%%%%%%%%%
% Uni Duesseldorf
% Lehrstuhl fuer Datenbanken und Informationssysteme
% Vorlage fuer Bachelor-/Masterarbeiten
% Optimiert fuer den Original-Latex-Kompiler LATEX.EXE (LaTeX=>PS=>PDF)
%%%%%%%%%%%%%%%%%%%%%%%%%%%%%%%%%%%%%%%%%%%%%%%%%%%%%%%%%%%%%%%%%%%%%%%%
% Ueberarbeitung für pdflatex (LaTeX=>PDF)
%%%%%%%%%%%%%%%%%%%%%%%%%%%%%%%%%%%%%%%%%%%%%%%%%%%%%%%%%%%%%%%%%%%%%%%%
% Vorlage Changelog:
% 10.09.2015 (Matthias Liebeck): Nummerierung des Inhaltsverzeichnis nun römisch, Beispiel für einen Anhang eingebaut, \raggedbottom hinter sections eingefügt
%%%%%%%%%%%%%%%%%%%%%%%%%%%%%%%%%%%%%%%%%%%%%%%%%%%%%%%%%%%%%%%%%%%%%%%%
%%%% BEGINN EINSTELLUNG FUER DIE ARBEIT. UNBEDINGT ERFORDERLICH! %%%%%%%
%%%%%%%%%%%%%%%%%%%%%%%%%%%%%%%%%%%%%%%%%%%%%%%%%%%%%%%%%%%%%%%%%%%%%%%%
% Geben Sie Ihren Namen hier an:
\newcommand{\bearbeiter}{Ole Kiefer}

% Geben Sie hier den Titel Ihrer Arbeit an:
\newcommand{\titel}{Collaborative Filtering}
\newcommand{\titeldeutsch}{Kollaboratives Filtern}



% Geben Sie das Datum des Beginns und Ende der Bachelorarbeit ein:
\newcommand{\beginndatum}{22. Januar 2016}
\newcommand{\abgabedatum}{10.~M\"{a}rz~2016}

% Geben Sie die Namen des Erst- und Zweitgutachters an:
\newcommand{\erstgutachter}{Prof. Dr.~Stefan Harmeling}
\newcommand{\zweitgutachter}{Prof. Dr.~Stefan Conrad}

% Falls Sie die Arbeit zweiseitig ausdrucken wollen,
% benutzen Sie die folgende Zeile mit
% \AN fuer zweiseitigen Druck
% \AUS fuer einseitigen Druck
\newcommand{\zweiseitig}{\AN}

% Falls die Arbeit in englischer Sprache verfasst 
% werden soll, dann benutzen Sie die folgende Zeile mit
% englisch fuer englische Sprache
% deutsch fuer deutsche Sprache
\newcommand{\sprache}{deutsch}

% Hier wird eingestellt, ob es sich bei der Arbeit um eine Bachelor- 
% oder Masterarbeit handelt (unpassendes auskommentieren!):
\newcommand{\arbeit}{Bachelorarbeit}
%~ \newcommand{\arbeit}{Masterarbeit}




%%%%%%%%%%%%%%%%%%%%%%%%%%%%%%%%%%%%%%%%%%%%%%%%%%%%%%%%%%%%%%%%%%%%%%%%
%%%% ENDE EINSTELLUNGEN %%%%%%%%%%%%%%%%%%%%%%%%%%%%%%%%%%%%%%%%%%%%%%%%
%%%%%%%%%%%%%%%%%%%%%%%%%%%%%%%%%%%%%%%%%%%%%%%%%%%%%%%%%%%%%%%%%%%%%%%%

% Die folgende Zeile NICHT EDITIEREN oder loeschen
\input{titelmakros}
%\pagenumbering{arabic}
%\setcounter{page}{1}



%%%%%%%%%%%%%%%%%%%%%%%%%%%%%%%%%%%%%%%%%%%%%%%%%%%%%%%%%%%%%%%%%%%%%%%%
%%%% BEGINN TEXTTEIL %%%%%%%%%%%%%%%%%%%%%%%%%%%%%%%%%%%%%%%%%%%%%%%%%%%
%%%%%%%%%%%%%%%%%%%%%%%%%%%%%%%%%%%%%%%%%%%%%%%%%%%%%%%%%%%%%%%%%%%%%%%%

%%%%%%%%%%%%%%%%%%%%%%%%%%%%%%%%%%%%%%%%%%%%%%%%%%%%%%%%%%%%%%%%%%%%%%%%
% Text entweder direkt hier hinein schreiben oder, im Sinne der
% besseren Uebersichtlich- und Bearbeitbarkeit mittels \input die
% einzelnen Textteile hier einbinden.
%%%%%%%%%%%%%%%%%%%%%%%%%%%%%%%%%%%%%%%%%%%%%%%%%%%%%%%%%%%%%%%%%%%%%%%%


\section{Einleitung}\label{s.Einleitung}\raggedbottom 
Jeder Homepagebetreiber hat das Ziel den Benutzer möglichst lange auf der eigenen Homepage zu halten, um möglichst viele Klicks und Werbeeinspielungen zu generieren. Ein Weg dies zu erreichen: personalisierte Webseiten, Werbung und Kaufvorschläge. Eine Online-Zeitung schlägt deshalb beim Lesen eines Artikels weitere Artikel vor, die einen interessieren könnten. Marketing-Firmen analysieren den Benutzer, um mit gezielter Werbung mehr Umsatz zu erzielen. Ein 20-jähriger Mann wird eher Autowerbung interessant finden als Make-Up-Werbung. Online-Shops geben Kaufvorschläge zu Artikeln, die man gerade ansieht oder in den Warenkorb gelegt hat, um weitere Käufe anzuregen. Online-Video- oder Musik-Plattformen schlagen ähnliche Filme oder Musikstücke vor, um den Nutzer zu halten.\\
Die Daten, die nötig sind, um den User zu analysieren und zu bewerten, liefert dieser meistens selbst. Tracker verfolgen jeden Klick, jede Mausbewegung wird mit Timern analysiert, Warenkörbe werden ausgewertet, Cookies speichern Daten für die nächste Sitzung. Daraus lassen sich viele Informationen ableiten, man kann abschätzen, wen man zu Besuch hat und womit man Aufmerksamkeit und letztendlich Klicks erreichen kann. In vielen Bereichen liefern die Benutzer ganz bewusst Informationen. Wenn sie etwas bewerten, ihr Profil ausfüllen oder auf den "`like"'-Button drücken. Hierbei treten jedoch zwei Probleme auf. Sehr wenige Nutzer bewerten regelmäßig und extreme Bewertungen (sehr schlecht oder sehr gut) kommen häufiger vor, als die differenzierten Zwischenstufen.\\
Um verschiedene Musikstücke genauer bewerten zu können, setzt der Musik-Streaming-Dienst PANDORA\footnote{\url{www.pandora.com}} deshalb auf ein Expertensystem. Im Music Genome Project\footnote{\url{http://www.pandora.com/about/mgp}} werden 450 charakteristische Merkmale in Liedern von trainierten Musikern bewertet und analysiert. Studierte Musiker geben hier qualitativ hochwertige Bewertungen ab, um die Musikstücke korrekt zu kategorisieren.\\
Beim kollaborativen Filtern nutzt man diese Informationen über die User und Items, um Interessen zu vergleichen, Ähnlichkeiten zu bewerten und Vorschläge zu erstellen.\\ In dieser Arbeit stelle ich sechs Algorithmen vor und teste diese am MovieLens Datensatz 100k (s.\autoref{s.Datenmenge} \nameref{s.Datenmenge}), um sie miteinander vergleichen zu können.

\clearpage


\section{Algorithmen zum kollaborativen Filtern}\label{s.Algorithmen}\raggedbottom
Beim kollaborativen Filtern ist die grundsätzliche Frage: Wie ähnlich sind sich zwei Objekte?
Mathematisch ausgedrückt fragt man nach der Distanz zweier Objekte. Ein Ansatz ist nach der Distanz zweier User zu fragen, der andere die Ähnlichkeit zweier Items zu untersuchen. Um diese Distanz zu berechnen, gibt es verschiedene Methoden.\\
Betrachtet man die Bewertungen als Matrix $R$ mit den Usern als Zeilen und den Items als Spalten, so wird die Ähnlichkeit zweier User $u$ und $v$ durch die Distanz ihrer Zeilenvektoren bewertet. Die Distanz zweier Items $i$ und $j$ ist der Abstand der Spaltenvektoren. $r_{u,i}$ ist der $i$-te Eintrag der $u$-ten Zeile von $R$ und damit die Bewertung des Users $u$ zum Film $i$. $\bar{r}^{user}_{u}$ ist seine durchschnittliche Bewertung. $\hat{r}_{u,i}$ ist die in den Bereich $[-1,1]\subset \mathbb{R}$ normierte Bewertung.  $R^{user}_{u}$ bezeichne, die Menge der Filme, die von $u$ bewertet wurden, $R^{item}_{i}$ die Menge der User, die Film $i$ bewertet haben. $\bar{r}^{item}_{i}$ ist die durchschnittliche Bewertung des Items $i$ aller User. 
\begin{equation}
\begin{aligned}	
R^{user}_{u}&:=\{i | r_{u,i}>0\} \qquad 
R^{item}_{i}:=\{u | r_{u,i}>0\}\\\\
\bar{r}^{user}_{u}&:= \dfrac{\sum\limits_{i \in R^{user}_{u}}r_{u,i}}{|R^{user}_{u}|} \qquad
\bar{r}^{item}_{i}:= \dfrac{\sum\limits_{u \in R^{item}_{i}}r_{u,i}}{|R^{item}_{i}|}\\\\
\mathrm{min}_{R} &= \min\limits_{(u,i)\in R} r_{u,i} \qquad
\mathrm{max}_{R} = \max\limits_{(u,i)\in R} r_{u,i} \\\\
\hat{r}_{u,i}&:= \dfrac{2(r_{u,i}-\mathrm{min}_{R})-(\mathrm{max}_{R}-\mathrm{min}_{R})}{(\mathrm{max}_{R}-\mathrm{min}_{R})}\\
\label{definition}
\end{aligned}
\end{equation}
Im speziellen Fall der MovieLens-Daten ist $\mathrm{min}_{R} = 1$ und $\mathrm{max}_{R} = 5$, die minimale bzw. maximale Bewertung eines Filmes. \\
Ein einfacher Weg einen neuen Vorschlag zu erzeugen, ist den ähnlichsten Nutzer zu User $u$ zu finden und ein Item vorzuschlagen, dass $u$ noch nicht bewertet hat. Um ein wenig mehr Informationen zu nutzen und unabhängiger von persönlichen Vorlieben zu werden, benutzt man nicht nur einen, sondern $k$ ähnliche User. Die Menge der $k$ nächsten Nachbarn $\mathrm{kNN}_{d}(u)$, sei gefüllt mit $k$ Usern mit dem geringsten Abstand, berechnet durch $d$, zu User $u$. Da der Test in \autoref{s.Test} (\nameref{s.Test}) das Rating für bestimmte Filme benötigt, werden in diesem Test nur Nachbarn in Betracht gezogen, die den Film $i$ auch bewertet haben. Damit wird sicher gestellt, dass die ausgewählten Nachbarn den Film bewertet haben.\\
Um ein Rating für einen Film zu erzeugen, kann man nun den Mittelwert der Bewertungen der $k$ nächsten Nachbarn bilden für diesen Film. 
\begin{equation}
r_{d}(u,i) := \dfrac{\sum\limits_{v \in \mathrm{kNN}_{d}(u)} r_{v,i}}{|\mathrm{kNN}_{d}(u)|}  
\label{rating}
\end{equation}
Diese Formel kann man als Grundformel für User-basierte Verfahren ansehen.\\\\
In den nächsten Abschnitten werden die verschiedenen Algorithmen beschrieben. Die euklidische Distanz dient als Einstieg in die Abstandsberechnung. Die Verfahren Pearson Correlation Coefficient \cite[Kap. 2, S.~23]{G2DM}, Adjusted Cosine Similarity \cite[Kap. 3, S.~16]{G2DM} und Slope One \cite[Kap. 3, S.~28]{G2DM} sind beschrieben in dem Buch "`A Programmer's Guide to Data Mining: The Ancient Art of the Numerati"' \cite{G2DM} von Ron Zacharski. Es folgt eine Idee aus der Graphentheorie. Mittels Floyd-Warshall werden kürzeste Wege zwischen Usern gesucht, um weitere Ähnlichkeiten zu errechnen. Den Abschluss bildet ein Hybrid-Verfahren, eine Kombination aus Euklid und Slope One.\\
Für alle Formeln und Algorithmen gilt die Einschränkung für $R^{user}_{u}\cap R^{user}_{v} \neq \emptyset$ bzw. $R^{item}_{i}\cap R^{item}_{j} \neq \emptyset$. Ist die Schnittmenge zwischen zwei Usern oder Items leer, wird keine Distanz zwischen diesem Paar definiert.

\subsection{User-basierte Algorithmen}
\subsubsection{Euklidische Distanz}\label{s.euclid}
Eine simple Metrik ist die euklidische Distanz:
	\begin{equation}
	\begin{aligned}
	d_{\mathrm{euclid}}(u,v) := \sqrt{\sum\limits_{i \in R^{user}_{u}\cap R^{user}_{v}} (r_{u,i}-r_{v,i})^2  }
	\qquad \mathrm{, f\ddot{u}r }\quad R^{user}_{u}\cap R^{user}_{v} \neq \emptyset
	\label{euclid}
	\end{aligned}
	\end{equation}

Ist die Schnittmenge zwischen User $u$ und User $v$ leer, so wird keine Distanz zwischen ihnen definiert.
Die euklidische Distanz zwischen zwei Benutzern $u$ und $v$ wird berechnet durch alle Items, die sowohl von User $u$ als auch von User $v$ bewertet wurden. Eine geringe Distanz suggeriert eine hohe Ähnlichkeit. Dies führt jedoch zu dem Problem, dass zwei Personen über die sehr wenig gemeinsame Informationen verfügen, ähnlicher bewertet werden können, als zwei Personen über die man sehr viele gemeinsame Informationen hat. Jede Information die man nutzt, vergrößert im Allgemeinen den Abstand zwischen zwei Objekten.\\
Eine Idee, dies zu verbessern, wäre eine Strafe einzubauen für Informationen, die man über den User $u$ kennt, aber über User $v$ nicht.
Ich habe mich dazu entschieden die Distanz zur mittleren Bewertung $(\dfrac{1}{2}(\mathrm{max}_{R}-\mathrm{min}_{R})+\mathrm{min}_{R})$ als Strafe einzubauen. Als Beispiel: User $u$ hat Film $i$ mit 5 bewertet, User $v$ hat keine Bewertung zu diesem Film abgegeben, so wird der Wert 2 als Fehler addiert.
\begin{equation}
\begin{aligned}
\qquad	d&_{\mathrm{euclidpenalized}}(u,v) := \\ &\sqrt{\sum\limits_{i \in R^{user}_{u}\cap R^{user}_{v}} (r_{u,i}-r_{v,i})^2  + \sum\limits_{i \in R^{user}_{u}\backslash R^{user}_{v}} (r_{u,i}-(\dfrac{1}{2}(\mathrm{max}_{R}-\mathrm{min}_{R})+\mathrm{min}_{R})^2) }
	\label{euclidpenalty}
\end{aligned}
\end{equation}

Eine weitere Möglichkeit Nachbarn zu suchen, die viele Informationen teilen, ist durch die Anzahl der Überschneidungen zu dividieren, um eine mittlere Distanz zwischen zwei Items zu ermitteln. Damit wird die euklidische Distanz zum mittleren, quadratischen Fehler.
\begin{equation}
d_{\mathrm{euclidnormalized}}(u,v) := \dfrac{\sqrt{\sum\limits_{i \in R^{user}_{u}\cap R^{user}_{v}} (r_{u,i}-r_{v,i})^2  }}{|R^{user}_{u}\cap R^{user}_{v}|}
\label{euclidmean}
\end{equation}
Eine Bewertung wird mittels folgender Formel berechnet.
\begin{equation}
r_{d_{\mathrm{euclid}}}(u,i) := \dfrac{\sum\limits_{v \in \mathrm{kNN}_{d_{\mathrm{euclid}}}(u)} r_{v,i}}{|\mathrm{kNN}_{d_{\mathrm{euclid}}}(u)|}  
\label{euklidrating}
\end{equation}

\subsubsection{Pearson Correlation Coefficient (PCC)}\label{s.pearson}
Der Pearson Algorithmus errechnet eine Ähnlichkeit zwischen allen User mit Hilfe des Pearson Korrelations Koeffizienten. Der Wert zwischen $[-1,1]\subset \mathbb{R}$ wird auch als Pearson Score bezeichnet. Wenn sich zwei User in ihren Bewertungen übereinstimmen, haben sie eine Pearson Score von +1. Sind beide User komplett verschieden, bekommen sie eine Score von -1. 
\begin{equation}
	d_{\mathrm{pearson}}(u,v) := \dfrac{\sum\limits_{i \in R^{user}_{u}\cap R^{user}_{v}} (r_{u,i}-\bar{r}^{user}_{u})(r_{v,i}-\bar{r}^{user}_{v})}  {\sqrt{\sum\limits_{i \in R^{user}_{u}\cap R^{user}_{v}}(r_{u,i}-\bar{r}^{user}_{u})^2}\sqrt{\sum\limits_{i \in R^{user}_{u}\cap R^{user}_{v}}(r_{v,i}-\bar{r}^{user}_{v})^2}}
	 	\label{pccformula}
\end{equation}
Von jeder Bewertung $r_{u,i}$ des Users $u$ für Item $i$, wird der Durchschnitt aller Bewertungen des Users $\bar{r}^{user}_{u}$ abgezogen. Dadurch können sich zwei User ähnlich sein, unabhängig davon ob, der eine am oberen Ende der Skala und der andere am unteren Ende der Skala bewertet.
Diese zwei User würden einen großen euklidischen Abstand haben und damit als sehr verschieden gelten. Hat man die Distanzen zwischen dem User $u$ und allen anderen Usern errechnet, findet man den ähnlichsten User $v$, indem man nach dem Pearson Score sortiert. Dementsprechend beinhaltet die Menge der $k$ nächsten Nachbarn die User, mit dem höchsten Score. Eine weitere Optimierung im PCC ist, dass die Bewertungen der Nachbarn mit dem Pearson-Score gewichtet gemittelt werden. Dies erzeugt eine Liste an Vorschlägen mit Items die $u$ gefallen könnten. \autoref{rating} wird mit der Gewichtung angepasst:
\begin{equation}
r_{d_{\mathrm{pearson}}}(u,i) := \dfrac{\sum\limits_{v \in \mathrm{kNN}_{d}(u)} r_{v,i}\cdot d_{\mathrm{pearson}}(u,v)}{\sum\limits_{v \in \mathrm{kNN}_{d}(u)}d_{\mathrm{pearson}}(u,v)}  
\label{pearsonrating}
\end{equation}

\subsubsection{Floyd-Warshall (FW)}\label{s.flowar}
Ermittelt man die Distanz zwischen allen Usern, so erhält man eine User-User-Matrix. Betrachtet man diese Matrix als Graphen kann man mit Graphen-Algorithmen Beziehungen zwischen Usern finden.\\
Die Gewichte des Graphen sind die Distanzen zwischen den einzelnen Userpaaren. Zur Anwendung kommt hier der Euklid-Algorithmus mit Strafe aus \autoref{euclidpenalty}. Dadurch entsteht zwischen einigen Usern ein gewichteter Pfad. User die keine Übereinstimmungen haben, bekommen die Distanz 1000.
	\begin{equation}
	d_{\mathrm{flowar}}(u,v) :=\left\{ \begin{array}{ll} d_{\mathrm{euclidpenalized}}(u,v) & \quad \mathrm{if}\textbf{ } R^{user}_{u}\cap R^{user}_{v} \neq \emptyset \\  1000 & \quad \mathrm{else}\end{array}\right.	\label{weights}
	\end{equation}
Floyd-Warshall berechnet nun zwischen allen Knoten den kürzesten Pfad. Der Pfad mit der kürzesten Gesamtlänge zwischen Knoten $u$ und $v$ wird als Distanz zwischen diesen Beiden aktualisiert. \\
\\
\begin{algorithm}[H]
	\mbox{Floyd-Warshall-Algorithmus}\\
	\ForEach{w in User}{
		\ForEach{u in User}{
			\ForEach{v in User}{
					d[u][v] = min ( d[u][v], d[u][w] + d[w][v] )	
				}
			}
		}
\end{algorithm}
Wie beim Euklid-Algorithmus werden die $k$ ähnlichsten User, mit der niedrigsten Distanz, betrachtet, um ein gemitteltes Rating zu erzeugen.
\begin{equation}
r_{d_{\mathrm{flowar}}}(u,i) := \dfrac{\sum\limits_{v \in \mathrm{kNN}_{d}(u)} r_{v,i}}{|\mathrm{kNN}_{d}(u)|}  
\label{flowarrating}
\end{equation}
Die Idee Algorithmen aus der Graphen-Theorie für das kollaborative Filtern zu nutzen wird in dem Paper "`Studying Recommendation Algorithms by Graph Analysis"'\cite{graph} von Mirza, Keller und Ramakrishnan näher erläutert. 



\subsection{Item-basierte Algorithmen}
\subsubsection{Adjusted Cosine Similarity (ACS)}\label{s.adjcos}
Adjusted Cosine Similarity ist ein Item-basierter Algorithmus. Statt die Ähnlichkeit zweier User zu berechnen, sucht ACS nach der Ähnlichkeit zweier Items.
\begin{equation}
 d_{\mathrm{acs}}(i,j) := \dfrac{\sum\limits_{u\in R^{item}_{i} \cap R^{item}_{j}} (r_{u,i}-\bar{r}^{user}_{u})(r_{u,j}-\bar{r}^{user}_{u})}  {\sqrt{\sum\limits_{u\in R^{item}_{i} \cap R^{item}_{j}}(r_{u,i}-\bar{r}^{user}_{u})^2}\sqrt{\sum\limits_{u\in R^{item}_{i} \cap R^{item}_{j}}(r_{u,j}-\bar{r}^{user}_{u})^2}} 	\label{adjcosformula1}
\end{equation}
Die Formel berechnet die Ähnlichkeit zwischen Item $i$ und $j$. Im Gegensatz zu PCC summiert ACS nicht über alle Items, die zwei User verbinden, sondern über alle User, die zwei Items verbinden.\\
Um mit diesen Ähnlichkeiten jetzt ein Rating zu erzeugen wird die folgende Funktion benötigt.
\begin{equation}
 \hat{r_\mathrm{acs}}(u,i) := \dfrac{\sum\limits_{j\in R^{user}_{u}} (d_{\mathrm{acs}}(i,j)\cdot \hat{r}_{u,j})}  {\sum\limits_{j\in R^{user}_{u}} (|d_{\mathrm{acs}}(i,j)|)} 	\label{adjcosformula2}
\end{equation}	
$j$ sind alle Items die bisher von $u$ bewertet wurden. $\hat{r}_{u,j}$ ist das normalisierte Rating im Wertebereich $[-1,1]\subset \mathbb{R}$. Das heißt, man betrachtet alle Items die bisher vom User $u$ bewertet wurden und multipliziert die Ähnlichkeit zu Item $i$. Dividiert durch die Summe aller Ähnlichkeiten.\\
Danach wird das normalisierte Rating wieder in den ursprünglichen, nichtnormalisierten Ratingbereich transformiert.
\begin{equation}
r_{\mathrm{acs}}(u,i) := \dfrac{1}{2}(\hat{r}_{u,i}+1)(\mathrm{max}_{R}-\mathrm{min}_{R})+\mathrm{min}_{R} 	\label{adjcosformula3}
\end{equation}	

\subsubsection{Slope One}\label{s.slopeone}
Slope One ist ebenfalls ein Item-basierter Algorithmus. Hier wird die Distanz zwischen zwei Items als durchschnittliche Abweichung aller Bewertungen definiert. Aus diesen erzeugt man dann ein Rating für das neue Item.
 
\begin{equation}
\mathrm{freq}(i,j) := \mathrm{card}(R^{item}_{i} \cap R^{item}_{j})=|R^{item}_{i} \cap R^{item}_{j}|\\
\end{equation}
\begin{equation}
 d_{\mathrm{slope1}}(i,j) := \sum\limits_{u\in R^{item}_{i} \cap R^{item}_{j}}\dfrac{r_{u,i}-r_{u,j}}  {\mathrm{freq}(i,j)} 	\label{deviation}
\end{equation}	
$\mathrm{freq}(i,j)$ berechnet die Anzahl der User, die sowohl $i$ als auch $j$ in ihren Bewertungen haben. $d_{\mathrm{slope1}}$ berechnet die Item-Item Matrix mit den Abweichungen zwischen allen Items. 
Um jetzt eine Vorhersage für den User $u$ für das bisher nicht von ihm bewertete Item $i$ machen zu können, kann man die vorher berechneten Abweichungen nutzen.
\begin{equation}
 r_{\mathrm{slope1}}(u,i) := \dfrac{\sum\limits_{j\in R^{user}_{u}}( d_{\mathrm{slope1}}(i,j)+r_{u,j})\mathrm{freq}(i,j)}  {\sum\limits_{j\in R^{user}_{u}}\mathrm{freq}(i,j)} 	\label{slopeone}
\end{equation}
Der Zähler bedeutet: Für jedes von User $u$ bewertete Item $j$ addieren wir zu $r_{u,j}$ die Abweichung $d_{\mathrm{slope1}}(i,j)$. Dies wird mit der Anzahl der User multipliziert, die beide Items $i$ und $j$ bewertet haben. Danach wird durch die Anzahl aller User geteilt, die sowohl Item $i$ als auch Items des Users $u$ mit einer Bewertung versehen haben. Die Abweichungen als auch die Anzahl der Bewertungen, die in die mittlere Abweichung einbezogen wurden, kann man vorweg berechnen und in Item-Item-Matrizen speichern. Eine neue Bewertung $r_{u,i}$ kann, durch die Frequenz-Matrix, sehr schnell in die Abweichungen hinzu gerechnet werden, ohne dass die komplette Matrix neu berechnet werden muss.
	\begin{equation}
\forall j\in R^{user}_{u}:\quad d_{\mathrm{slope1}}(i,j) := \dfrac{(d_{\mathrm{slope1}}(i,j)\mathrm{freq}(i,j)+(r_{u,i}-r_{u,j}))}  {\mathrm{freq}(i,j)+1} 	\label{slopeoneadd}
	\end{equation}

\subsection{Hybrid}\label{s.hybrid}
Hybrid ist ein Mix aus dem Verfahren des Euklidischen Abstandes mit Strafe und SlopeOne. Es wird also einmal eine Bewertung über die User-Ähnlichkeit und einmal über die Item-Ähnlichkeit berechnet und diese beiden Bewertungen werden gemittelt. Ein Parameter $a$ entscheidet, wie stark die Ratings der beiden Algorithmen ins Gewicht fallen.
\begin{equation}
r_{\mathrm{hybrid}}(u,i) := (1-a)r_{d_{\mathrm{euclidpenalized}}}(u,i) + a r_{\mathrm{slope1}}(u,i)  	\label{hybridrating}
\end{equation}
Dieser Algorithmus ist ein Versuch, wie stark eine Kombination aus zwei Methoden sein kann. 
	
\begin{table}[t!]
	\renewcommand{\arraystretch}{1.5}
	\centering
	\caption{Liste der Algorithmen}
	\begin{tabular}{c c}
		\multirow{3}*{User-basiert} & \nameref{s.euclid} \\
		& \nameref{s.pearson} \\& \nameref{s.flowar} \\ \\
		& \nameref{s.hybrid} \\ \\
		 \multirow{2}*{Item-basiert} & \nameref{s.adjcos} \\& \nameref{s.slopeone}\\ 
	\end{tabular}
	\label{tab:Liste der Algorithmen}
	\renewcommand{\arraystretch}{1}
\end{table}

\clearpage

\section{Vorhersage zu der Filmdatenbank MovieLens}\label{s.Test} \raggedbottom
Um die sechs Algorithmen miteinander vergleichen zu können, treten sie in verschiedenen Szenarien (s.  \autoref{s.split} \nameref{s.split}) gegeneinander an. Als Datenbank dient das MovieLens Dataset 100k. Als zweiten Test habe ich eine niedrig dimensionierte, randomisierte Matrix erstellt. Die Frage ist, ob man parallelen zwischen den beiden Datenmengen erkennt und dieses Wissen nutzen kann.

\subsection{Filmdatenbank Movielens}\label{s.Datenmenge}
Der verwendete Datensatz, auf dem die Algorithmen laufen, ist der MovieLens Datensatz 100k\footnote{\url{http://grouplens.org/datasets/movielens/}}. Dieser ist frei verfügbar und ist seit 1998 unverändert online, um erzielte Ergebnisse miteinander vergleichen zu können. Die Datensätze sind durch Userbewertungen, die auf Movielens\footnote{\url{https://movielens.org/}} abgegeben wurden, entstanden.
Im 100k Datensatz sind folgende Daten enthalten:
\begin{itemize}
	\item 943 User
	\item 1682 Filme
	\item 100.000 Ratings
	\item Ratings von 1-5
	\item Jeder User hat mindestens 20 Filme bewertet
	\item Demographische Informationen wie Geschlecht, Alter, Wohnort
\end{itemize}
Im Laufe der Jahre sind weitere, größere Datensätze entstanden. 
Ebenfalls verfügbar sind die Datensätze 1M mit einer Million Bewertungen von 6000 User zu 4000 Filmen. Im Jahr 2009 ist der 10M Datensatz zur Verfügung gestellt worden. Zusätzlich zu den 72.000 User und 10.000 Filmen wurden noch 100.000 Tags der Datenbank hinzugefügt. Diese Informationen helfen die Vorschläge stetig zu verbessern und zu optimieren. Im Jahr 2015 hat sich die Datenbank noch einmal verdoppelt auf 20M (27k Filme / 138k User / 100k Tags). Weiterhin gibt es zwei Datensätze die sich regelmäßig verändern und somit ständig neue Voraussetzungen bieten.


\subsection{Aufteilung der Datenmenge in Test- und Trainingsdaten}\label{s.split}
Die Datenmenge wurde aufgeteilt in Test- und Trainingsdaten. In den Trainingsdaten sind die Informationen enthalten, die der Algorithmus nutzen kann um Vorhersagen zu generieren. Die Testdaten werden genutzt um die Vorhersagen zu überprüfen mittels der echten Bewertungen der User. Jeder User hat mindestens 20 Filme bewertet. Um verschiedene Szenarien testen zu können werden $l$ Bewertungen pro User aus der Datenmenge in die Testmenge geschoben, mit $l \in \{1,5,10,19\}$. Alle anderen Bewertungen befinden sich in den Trainingsdaten.\\
Für das Test-Szenario $l = 1$ haben alle User noch mindestens 19 Bewertungen in den Trainingsdaten. Das heißt, dass sehr viele Informationen zur Verfügung stehen. Dies beeinflusst die Wahl ähnlicher User und natürlich das Wissen über einen bestimmten User. Wie wertet er im Mittel? Welche Genres findet er gut oder schlecht? Im Fall $l = 19$ haben einige Nutzer nur noch einen Film in der Trainingsmenge. In diesem Szenario wird es sehr viel schwieriger eine passende Vorhersage zu ermitteln.
	
\subsection{Matrix mit niedrigem Rang}
Als zweiter Testdatensatz dient eine zufällig erstellte Matrix mit niedrigem Rang. Die Frage, die sich hier stellt ist, kann man die Datenmenge auf eine Basis von typischen User reduzieren. Wenn man solche User finden kann, kann man mit weiteren Verfahren eine geeignete Linearkombination für jeden User finden, um approximativ die fehlenden Einträge der Matrix zu berechnen. In dieser Arbeit teste ich die oben genannten Algorithmen an einer solchen Menge, um eventuell Parallelen zwischen den beiden Datensätzen zu erkennen. Die Testmenge besteht aus 30 typischen, linear unabhängigen Usern die alle 1682 Filme zwischen 1 und 5 bewertet haben. Diese Matrix wird per Linearkombination der 30 Basis-User auf 943 User erweitert, mit einem Rang 30.
\begin{algorithm}[H]
o = 30 \\
m = 943 \\
n = 1682 \\
A1 = random.uniform(1,5,(m,o))\\
A2 = random.uniform(1,5,(o,n))\\
B = A1$^{T}$A2
\end{algorithm}
Matrix $B$ hat jetzt eine Größe von $943x1682$ und Rang 30. Um die Matrix an die MovieLens-Daten anzupassen, habe ich die Einträge in den Bereich 1 bis 5 transformiert und zufällig 93,7\% der Bewertungen entfernt, um die gleiche Dichte wie in den MovieLens-Daten (6,3\%) zu erreichen. Mit diesem Datensatz wird nur das Szenario $l = 1$ untersucht. Da der ganze Vorgang zufällig ausgewählt wird, ist in diesem Szenario nicht garantiert, dass über jeden User noch genug Informationen durch Bewertungen in der Trainingsmenge vorliegen.

\subsection{Berechnung des Fehlers}\label{s.error}
Die Güte der Algorithmen wird über den durchschnittlichen absoluten Fehler (Mean Absolute Error) ermittelt. Jede User-Item-Bewertung aus der Testmenge $R^{test}$ wird der Vorhersage des Algorithmus gegenüber gestellt. Der absolute Fehler wird aufsummiert und durch die Anzahl der Datensätze in der Testmenge dividiert.
\begin{equation}
		\mathrm{MAE} := \dfrac{\sum\limits_{(u,i)\in R^{test}}|r(u,i)-r^{test}_{u,i}|}{\mathrm{card}(R^{test})}  	\label{MAE}
\end{equation}
Ein MAE von 1 bedeutet demnach, dass der Algorithmus mit seinem Vorschlag um durchschnittlich einen Stern daneben liegt. Bei einem MAE von 0 ist die Testmenge perfekt getroffen. Die Bewertungen in der Datenmenge liegen im Ganzzahlbereich ${1,2,3,4,5}$. Die Algorithmen erstellen Ratings im Bereich $[1,5]\subset \mathbb{R}$. Dies führt zu einem Wertebereich des MAE von $[0,4]\subset \mathbb{R}$.

\clearpage
\section{Vergleich verschiedener Algorithmen}\label{s.Ergebnisse}\raggedbottom

\subsection{Wahl der Parameter}

\subsubsection{Vergleich der drei Euklid Varianten}\label{s.keuclid}
\begin{figure}[htbp!]
	\centering
	\includegraphics[width=1\linewidth]{ErrorEuclid}
	\caption{MAE im Euklid Algorithmus in Abhängigkeit der Parameter}
	\label{fig:ErrorEuclid}
	
\end{figure}
Die Werte sind mit einer Farbskala von Rot über Gelb nach Grün je Zeile formatiert, um den maximalen Fehler (rot) und den minimalen Fehler (grün) besser erkennen zu können.\\\\
Wie man an den Tabellen sehen kann, verbessert die Normalisierung den Standard-Euklid-Algorithmus aus \autoref{euclid} . Eine stärkere Verbesserung wird jedoch durch die eingebaute Strafe erzielt. Mit den Parametern $k = 15$ und der Strafe kann der Fehler minimiert werden. In den folgenden Vergleichen werden immer diese Parameter benutzt.
\clearpage

\subsubsection{Wahl von $k$ bei den kNN im Pearson Algorithmus}\label{s.kpearson}
\begin{figure}[htbp!]
\centering
\includegraphics[width=1\linewidth]{ErrorPearsonk}
\caption{MAE im Pearson Algorithmus in Abhängigkeit der Parameter }
\label{fig:ErrorPearsonk}
\end{figure}
Eine Wahl von $k = 10$ nächsten Nachbarn optimiert den Pearson Algorithmus. Im Szenario $l = 1$ zeigt der Algorithmus seine Stärken. Wenn viele Informationen vorhanden sind, können mit dem Pearson Algorithmus gute Bewertungen vorhergesagt werden. Wenn man die mittleren Szenarien $l = 5$ und $l = 10$ mit $l = 19$ vergleicht, fällt auf, dass das letzte Szenario am besten abschneidet. Da hier 19 Fehler berechnet und gemittelt werden. Im Fall $l = 5$ werden nur fünf Fälle verglichen, dadurch entsteht im Mittel eine größere Abweichung zwischen der Vorhersage und der tatsächlich abgegeben Bewertung.
\clearpage

\subsubsection{Wahl von $k$ bei den kNN im Floyd Warshall Algorithmus}\label{s.kflowar}
\begin{figure}[htbp!]
\centering
\includegraphics[width=1\linewidth]{ErrorFloWark}
\caption{MAE im Floyd Warshall Algorithmus in Abhängigkeit der Parameter}
\label{fig:ErrorFloWark}
\end{figure}
Man sieht in \autoref{fig:ErrorFloWark}, dass Floyd-Warshall bei großem $k$ (=200-500) am besten funktioniert. Es wird das Minimum aus den $k$ nächsten Nachbarn und den n nächsten Nachbarn, die den aktuell zu bewertenden Film überhaupt bewertet haben, genommen, um das zu erwartete Rating zu ermitteln. Dies tendiert gegen den Durchschnittswert aller User, die die fehlenden Items bewertet haben. Somit ist der FW-Algorithmus fast gleichzusetzen mit dem Itemmean.
\clearpage

\subsubsection{Wahl von $a$ beim LUS-GUS Algorithmus}\label{s.alugu}

\begin{figure}[htbp!]
\centering
\includegraphics[width=1\linewidth]{ErrorLUSGUSa}
\caption{MAE im LUS-GUS Algorithmus in Abhängigkeit der Parameter}
\label{fig:ErrorLUSGUSa}
\end{figure}
\FloatBarrier
Sind viele Informationen über die User gegeben ( $l = 1$ ) kann der LUS-GUS die Stärken des Pearson Algorithmus ausspielen. Es finden sich gute direkte Nachbarn die sehr ähnlich sind. Trotzdem kann das Wissen aus dem Floyd-Warshall-Algorithmus über indirekte Nachbarn den MAE weiter minimieren. Optimal wird das Rating zu 80\% aus dem FW und zu 20\% aus dem Pearson Algorithmus ermittelt. Ist nicht viel über die User bekannt, wird der Graph-Algorithmus bedeutender. Informationen die aus einer Kombination aus mehreren User entstehen, erzielen in diesem Fall bessere Ergebnisse.


\subsection{Auswertung der Fehlerberechnung}\label{s.auswertung}
Neben den 6 genannten Algorithmen habe ich noch 3 weitere Algorithmen für den Vergleichstest implementiert. Der einfachste, Random, erzeugt einen randomisierten Float zwischen $[1,5]\subset \mathbb{R}$. Usermean $\bar{R}^{user}_{u}$ kalkuliert den Durchschnitt des Users, der gerade betrachtet wird. Itemmean $\bar{R}^{item}_{i}$ berechnet die durchschnittliche Bewertung aller User des Items, das gerade im Fokus ist (s. \autoref{definition}). 
Gegen diese 3 simplen Algorithmen werden die MAE der 6 oben beschriebenen Funktionen gemessen. Wobei wieder alle 4 Szenarien bezüglich des Parameters $l$ betrachtet werden.\\

\begin{figure}[htbp!]
\centering
\includegraphics[width=1\linewidth]{Vergleich}
\caption{MAE Vergleich aller Algorithmen}
\label{fig:Vergleich}
\end{figure}
\FloatBarrier
Die Laufzeiten dienen als relativer Vergleich zwischen den Algorithmen. Die Daten sind in einer Hashmap gespeichert und es wird nur ein Kern der CPU ausgelastet. Durch andere Implementierungen und andere PC-Setups können die Zeiten variieren.\\
Usermean und Itemmean erzielen im Vergleich kein schlechtes Ergebnis, zählen sogar laut MAE mit zu den besten Algorithmen. Sie sind aber für Vorhersagen nicht verwendbar. Bei Usermean bekommt jedes unbekannte Item das gleiche Rating. So können keine Vorschläge gemacht werden, die aus dem besten Rating resultieren. Bei Itemmean werden die Vorlieben des Users komplett ignoriert. Pearson erkennt sogar Ähnlichkeiten in relativen Abweichungen und kommt somit mit einem geringem $k$ im k-Nearest-Neighbors aus. Das genaue Gegenteil ist bei Floyd-Warshall der Fall. Der sehr groß eingestellte Parameter $k$ führt dazu, dass FW ähnlich wie Itemmean die bekannten Informationen über den User vernachlässigt. Da mit dem Itemmean eine bessere durchschnittliche Abweichung erzielt wird, als über die entdeckten Pfade zu anderen User. Dennoch kann FW den Pearson Algorithmus verbessern. Die Kombination beider Algorithmen im LUSGUS-Verfahren erzielt leicht bessere Ergebnisse, ist aber in meinem Test den Mehraufwand an Rechenleistung, gegenüber Pearson, nicht Wert. Adjusted Cosine Similarity ist in diesem Test nicht sehr erfolgreich. Obwohl es mit der gleichen Score arbeitet wie der PCC und dort alle bekannten Informationen über die zwei Items verwendet, können keine guten Vorhersagen erzielt werden. Der Algorithmus mit dem euklidischen Abstand und Strafe schneidet sehr gut ab und hat dazu sehr kurze Berechnungszeiten. Bessere Vorhersagen erzielt man nur mit dem Item-basierten Algorithmus Slope One. Die absolute durchschnittliche Differenz zwischen den Items verwendet alle verfügbaren Informationen aus dem Datensatz. Neben den Differenzen zwischen allen Items muss auch noch eine Frequenzmatrix gespeichert werden. Neue Informationen können dadurch ohne großen Rechenaufwand in die Abstandsmatrix eingearbeitet werden. Die Vorberechnungen im SlopeOne machen ca. 30 Sekunden der gesamten Zeit aus.\\
Die Entscheidung welchen Algorithmus man verwenden sollte liegt an vielen Faktoren. Hat man sehr viele User auf wenige Items, trumpfen Item-basierte Algorithmen auf. Die Item-Item-Matrix hat in diesem Fall kleinere Dimensionen. User-basierte Verfahren müssen erst die Abstände zu allen anderen Nutzern berechnen um die ähnlichsten Nutzer zu finden. Ist das Verhältnis User zu Items anders herum, können User-basierte Algorithmen eine Alternative sein. Alle User zu betrachten ist unter diesen Umständen eventuell schneller oder nicht so speicherintensiv wie eine sehr große Item-Item-Matrix. Generell wird eine Kombination aus mehreren Verfahren in der Praxis die besten Ergebnisse erzielen. Demographische Informationen aus dem Userprofil oder Tags bieten weitere Ansätze um Distanzen zwischen User oder Items zu erstellen. \\
Zudem haben alle Algorithmen Schwierigkeiten damit einem neuem User Vorschläge zu bereiten, da sein persönlicher Geschmack noch nicht bekannt ist. Neue Items zu integrieren erfordert ebenfalls andere Ansätze.
\begin{figure}[htbp!]
	\centering
	\includegraphics[width=1\linewidth]{bilder/lowrank}
	\caption{MAE Vergleich mit einer Matrix mit niedrigem Rang}
	\label{fig:LowRankMatrix}
\end{figure}
\FloatBarrier
Bei der Konstruktion mit niedrigem Rang erkennt man, dass die User-basierten Algorithmen richtig gut werden. Das ist aber auch zu erwarten, denn die User ähneln sich sehr stark, da sie alle aus einer Kombination aus dreißig Basis-User entstanden sind. In diesem Szenario kann das Euklid-Verfahren den sonst starken Slope-One von der Top Position verdrängen. Die Bewertungen wurden zufällig erstellt, dadurch können auch keine Ähnlichkeiten zwischen den Filmen entstehen. Dies wertet die Item-basierten Algorithmen zusätzlich ab.\\
Einen Vergleich zu der realen Datenbank von MovieLens kann man nur mit Vorsicht tätigen. Wenn man die Histogramme der beiden Matrizen betrachtet, wird klar, dass die Konstruktion der Matrix mit niedrigem Rang nicht sehr nah an die Verteilung der MovieLens-Daten heran kommt.
\begin{figure}[htbp!]
	\centering
	\includegraphics[width=1\linewidth]{histogram}
	\caption{Histogramm Vergleich}
	\label{fig:histogram}
\end{figure}
\FloatBarrier
Beim Versuch die Bewertungen gleichmäßiger zu verteilen zerstört man die lineare Abhängigkeit der Zeilen.

\begin{figure}[htbp!]
	\centering
	\includegraphics[width=1\linewidth]{lowranknotnorm}
	\caption{MAE Vergleich ohne Normalisierung}
	\label{fig:LowRankMatrixnotnorm}
\end{figure}
\FloatBarrier
In den Daten in \autoref{fig:LowRankMatrixnotnorm} wurde bewusst vermieden, die Bewertungen wieder in den Wertebereich $[1,5]\subset \mathbb{N}$ zu normieren. Dadurch wird die lineare Abhängigkeit der Zeilen nicht zerstört. Diese erzielten Fehler können deshalb nicht mit den anderen Szenarien verglichen werden. Was man aber erkennen kann, dass die User-basierten Verfahren sehr viel besser abschneiden, als der Item-basierte Algorithmus Adjosted Cosine Similarity. Slope One kann auch hier mit den vollen Informationen der Matrix ein sehr gutes Ergebnis erzielen.\\
Abschließend lässt sich keine große Aussage mit diesem zweiten Test erzielen. Die User-basierten Algorithmen sind per Konstruktion sehr gut. Die Item-basierten Verfahren können durch die zufälligen Bewertungen keine Ähnlichkeiten zwischen den Items erkennen. Zusätzlich ist die Verteilung in der Konstruktion weit entfernt von der realen Verteilung der MovieLens-Daten.

\clearpage
\section{Schlussfolgerung}
Die Entscheidung, welchen Algorithmus man verwenden sollte, liegt an vielen Faktoren. Hat man sehr viele User auf wenige Items, trumpfen Item-basierte Algorithmen auf. Die Item-Item-Matrix der Distanzen zwischen zwei Items hat in diesem Fall kleinere Dimensionen. User-basierte Verfahren müssen erst die Abstände zu allen anderen Nutzern berechnen, um die ähnlichsten Nutzer zu finden. Ist das Verhältnis User zu Items anders herum, können User-basierte Algorithmen schneller sein. Alle User zu betrachten ist unter diesen Umständen eventuell schneller oder nicht so speicherintensiv wie eine sehr große Item-Item-Matrix. Generell wird eine Kombination aus mehreren Verfahren in der Praxis die besten Ergebnisse erzielen. Der Hybrid Algorithmus erzielt in diesem Test durchweg die besten Ergebnisse. Weitere Kombinationen sind durchaus vorstellbar, wie eine Linearkombination aus mehreren Verfahren, oder Abfragen ob viel oder wenig über den User bekannt ist und dementsprechend die Stärken der Algorithmen gezielt auszunutzen.
\\
Zudem haben alle Algorithmen Schwierigkeiten damit, einem neuem User Vorschläge zu bereiten, da sein persönlicher Geschmack noch nicht bekannt ist. Hier kann der Itemmean seinen Nutzen zeigen, dieser kann allgemein beliebte Items vorschlagen. Eine erfolgreiche Methode, um neue Items zu integrieren, könnte sein über Tags, die z.B. Vertreter des gleichen Genres verlinken, die Items zu Klassifizieren. Vor allem professionell erstellte Tags haben enormes Potenzial, ähnliche Filme auf verschiedene Eigenschaften zu kategorisieren. Das erfordert eine Art Expertensystem, wie Pandora es nutzt.
\\
Demographische Informationen aus dem Userprofil bieten weitere Informationen um Ähnlichkeiten zwischen Usern zu erkennen. 
\\
Ein überraschendes Ergebnis, in diesem Test, konnte der Algorithmus mit dem euklidischem Abstand erzielen, ein simples Verfahren, dass leicht zu verstehen ist. Mit der Strafe als Anpassung kann der Fehler sehr gut minimiert werden. In Verbindung mit der kurzen Laufzeit hat diese Methode eine sehr gute Gesamt-Performance. Die aufwändigeren Verfahren brachten keine Verbesserung in der Fehlerminimierung. Das hybride Modell minimiert weiter den Fehler.
\\
Die Idee aus der Graphentheorie, kürzeste Wege über mehrere ähnliche Nachbarn zu gehen, um weiter entfernte Nachbarn näher zu bringen, hat in diesem Szenario nicht funktioniert. Die Fehlerminimierung ist nicht sehr gut und der Parameter $k$ ist, im besten Fall, viel zu groß. Damit sind die meisten, erzeugten Ratings gleich zu setzen mit dem Itemmean.
\\
Abschließend lässt sich keine große Aussage aus dem zweiten Test mit der zufällig erstellten Matrix erzielen. Die User-basierten Algorithmen sind per Konstruktion sehr gut. Die Item-basierten Verfahren können durch die zufälligen Bewertungen keine Ähnlichkeiten zwischen den Items erkennen. Zusätzlich ist die Verteilung in der Konstruktion weit entfernt von der realen Verteilung der MovieLens-Daten.


%%%%%%%%%%%%%%%%%%%%%%%%%%%%%%%%%%%%%%%%%%%%%%%%%%%%%%%%%%%%%%%%%%%%%%%%
%%%% ENDE TEXTTEIL %%%%%%%%%%%%%%%%%%%%%%%%%%%%%%%%%%%%%%%%%%%%%%%%%%%%%
%%%%%%%%%%%%%%%%%%%%%%%%%%%%%%%%%%%%%%%%%%%%%%%%%%%%%%%%%%%%%%%%%%%%%%%%

\clearpage

% Entfernen Sie das Kommentar aus der nachfolgenden Zeile, falls Sie einen Anhang in der Arbeit verwenden wollen. Beachten Sie, dass Sie sich im Verlauf der Arbeit mit \ref{...} (z.B. \ref{anhang:zusatz1}) auf den Anhang beziehen.
\newpage
\appendix
\section{Anhang: Quellcode}

Da ich vor Beginn dieser Arbeit noch nicht mit Python gearbeitet habe und mit der Syntax noch nicht vertraut war, habe ich mich dazu entschlossen als Start meines Quellcodes, die Recommender Klasse von Ron Zacharski zu nutzen. In seinem Buch "`A Programmer's Guide to Data Mining: The Ancient Art of the Numerati"' \cite{G2DM} stellt er neben den Erklärungen auch Python Quellcode \footnote{\url{https://github.com/zacharski/pg2dm-python}} zur Verfügung. In Chapter 3 befindet sich die Python Datei "recommender3.py". Darin enthalten sind der Import der Movie-Lens-Daten, der Pearson und der Slope One Algorithmus. Des weiteren befindet sich die Funktion computeSimilarity(band1, band2, userRatings) in der Datei "cosineSimilarity.py". Diese Funktion war meine Ausgangslage, um das Verfahren Adjusted Cosine Similarity zu implementieren.\\
Den Code habe ich an meine Bedürfnisse angepasst, damit ich den automatisierten Test mit den Testdaten ausführen konnte. Für diese Zwecke habe ich die Testklasse erstellt. Mit der Funktion $\mathrm{testit}(\mathrm{self}, \mathrm{fct}, k=10, a=0.5)$ kann man den Testlauf für eine der Verfahren starten. Der Parameter $k$ entscheidet die Anzahl der Nachbarn, $a$ wird als Parameter für den Hybrid Modus benötigt. 
\begin{equation}
\begin{aligned}
\mathrm{fct} \in \{\mathrm{'random', 'itemmean', 'usermean', 'euclid', 'pearson',}\\ \mathrm{'slopeone', 'floydwarshall', 'lusgus', 'adjcos'}\}
\end{aligned}
\end{equation}
Der Quellcode und der MovieLens-Datensatz ist online in meinem Github Repository \footnote{\url{https://github.com/kralle71/Collaborative-Filtering}} zu finden.


\subsection{Übersicht aller Funktionen} \label{anhang:übersicht}
\lstset{%
	breaklines=true
}
\lstinputlisting[language=Python]{CollaborativeFiltering-Summary.py}

\clearpage

\subsection{Quellcode} \label{anhang:quellcode}

\lstinputlisting[language=Python]{CollaborativeFiltering.py}


\clearpage

\bibliography{references}
\bibliographystyle{alphadin}
%\vspace*{\fill}

\clearpage

\listoffigures

\listoftables



%\pagebreak

%\printindex
\end{document}
