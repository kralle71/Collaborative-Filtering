%%% Die folgende Zeile nicht ändern!
\section*{\ifthenelse{\equal{\sprache}{deutsch}}{Zusammenfassung}{Abstract}}
%%% Zusammenfassung:
Beim kollaborativen Filtern versucht man die meist wenigen Informationen einer dünn besetzten Matrix zu verwenden, um diese sinnvoll aufzufüllen. In der Anwendung wird dieses Verfahren beispielsweise benutzt um Vorschläge für einen Nutzer zu kreieren oder Vorlieben eines Nutzers heraus zu finden. Dazu werden Distanzen zwischen den Zeilen oder Spalten der Matrix berechnet um ähnliche Einträge zu finden und neue Informationen zu gewinnen.\\
In dieser Arbeit stelle ich einige gängige Verfahren des kollaborativen Filterns vor, wie den Pearson Correlation Coefficient, Adjusted Cosine Similarity oder den Slope One Algorithmus. Mit dem Floyd-Warshall-Algorithmus wird eine Idee aus der Graphen-Analyse mit diesem Thema in Verbindung gebracht. Zusätzlich habe ich mir zwei Gedanken gemacht, wie ich die simple Metrik des euklidischen Abstandes verbessern kann, um bessere Ergebnisse mit diesem Algorithmus zu erzielen.\\
Der Vergleichstest arbeitet auf der frei verfügbaren MovieLens-Datenbank 100k, in der 943 User 1682 Filme bewertet haben. Die Datenbank wurde aufgeteilt in eine Trainings- und eine Testmenge. Die Algorithmen berechnen mögliche Filmbewertungen und werden mit der Testmenge verglichen. Der durchschnittliche Fehler dient als Vergleichsgröße wie gut der Algorithmus funktioniert. Zusätzlich vergleiche ich die Verfahren auf einer großen Matrix mit niedrigem Rang, in der viele Zeilen eine Linearkombination einer kleinen Menge linear unabhängiger User sind.\\
Im Kapitel 4 wird erst die Parameterwahl einiger Algorithmen getestet. Die produzierten Fehler werden miteinander verglichen und die Ergebnisse interpretiert, zum Beispiel in welchem Szenario sich welcher Algorithmus eignet. Überraschenderweise kann die Methode mit dem euklidischen Abstand mit Strafe ebenfalls sehr gute Ergebnisse erzielen und schlägt alle Algorithmen außer Slope One, welcher in diesem Szenario die besten Ergebnisse erzielt.\\
Im Anhang ist mein Quellcode zur Auswertung der Datenbank mittels der untersuchten Algorithmen in Python zu finden. In der Recommender Klasse sind alle Algorithmen implementiert. Die Testklasse dient als eine Art Toolbox um den Fehler zur Testmenge zu berechnen.
