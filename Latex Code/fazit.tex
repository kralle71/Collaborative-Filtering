\section{Schlussfolgerung}
Die Entscheidung, welchen Algorithmus man verwenden sollte, liegt an vielen Faktoren. Hat man sehr viele User auf wenige Items, trumpfen Item-basierte Algorithmen auf. Die Item-Item-Matrix der Distanzen zwischen zwei Items hat in diesem Fall kleinere Dimensionen. User-basierte Verfahren müssen erst die Abstände zu allen anderen Nutzern berechnen, um die ähnlichsten Nutzer zu finden. Ist das Verhältnis User zu Items anders herum, können User-basierte Algorithmen schneller sein. Alle User zu betrachten ist unter diesen Umständen eventuell schneller oder nicht so speicherintensiv wie eine sehr große Item-Item-Matrix. Generell wird eine Kombination aus mehreren Verfahren in der Praxis die besten Ergebnisse erzielen. Der Hybrid Algorithmus erzielt in diesem Test durchweg die besten Ergebnisse. Weitere Kombinationen sind durchaus vorstellbar, wie eine Linearkombination aus mehreren Verfahren, oder Abfragen ob viel oder wenig über den User bekannt ist und dementsprechend die Stärken der Algorithmen gezielt auszunutzen.
\\
Zudem haben alle Algorithmen Schwierigkeiten damit, einem neuem User Vorschläge zu bereiten, da sein persönlicher Geschmack noch nicht bekannt ist. Hier kann der Itemmean seinen Nutzen zeigen, dieser kann allgemein beliebte Items vorschlagen. Eine erfolgreiche Methode, um neue Items zu integrieren, könnte sein über Tags, die z.B. Vertreter des gleichen Genres verlinken, die Items zu Klassifizieren. Vor allem professionell erstellte Tags haben enormes Potenzial, ähnliche Filme auf verschiedene Eigenschaften zu kategorisieren. Das erfordert eine Art Expertensystem, wie Pandora es nutzt.
\\
Demographische Informationen aus dem Userprofil bieten weitere Informationen um Ähnlichkeiten zwischen Usern zu erkennen. 
\\
Ein überraschendes Ergebnis, in diesem Test, konnte der Algorithmus mit dem euklidischem Abstand erzielen, ein simples Verfahren, dass leicht zu verstehen ist. Mit der Strafe als Anpassung kann der Fehler sehr gut minimiert werden. In Verbindung mit der kurzen Laufzeit hat diese Methode eine sehr gute Gesamt-Performance. Die aufwändigeren Verfahren brachten keine Verbesserung in der Fehlerminimierung. Das hybride Modell minimiert weiter den Fehler.
\\
Die Idee aus der Graphentheorie, kürzeste Wege über mehrere ähnliche Nachbarn zu gehen, um weiter entfernte Nachbarn näher zu bringen, hat in diesem Szenario nicht funktioniert. Die Fehlerminimierung ist nicht sehr gut und der Parameter $k$ ist, im besten Fall, viel zu groß. Damit sind die meisten, erzeugten Ratings gleich zu setzen mit dem Itemmean.
\\
Abschließend lässt sich keine große Aussage aus dem zweiten Test mit der zufällig erstellten Matrix erzielen. Die User-basierten Algorithmen sind per Konstruktion sehr gut. Die Item-basierten Verfahren können durch die zufälligen Bewertungen keine Ähnlichkeiten zwischen den Items erkennen. Zusätzlich ist die Verteilung in der Konstruktion weit entfernt von der realen Verteilung der MovieLens-Daten.