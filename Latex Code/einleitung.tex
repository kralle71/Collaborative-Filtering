\section{Einleitung}\label{s.Einleitung}\raggedbottom 
Jeder Homepagebetreiber hat das Ziel den Benutzer möglichst lange auf der eigenen Homepage zu halten, um möglichst viele Klicks und Werbeeinspielungen zu generieren. Ein Weg dies zu erreichen: personalisierte Webseiten, Werbung und Kaufvorschläge. Eine Online-Zeitung schlägt deshalb beim Lesen eines Artikels weitere Artikel vor, die einen interessieren könnten. Marketing-Firmen analysieren den Benutzer, um mit gezielter Werbung mehr Umsatz zu erzielen. Ein 20-jähriger Mann wird eher Autowerbung interessant finden als Make-Up-Werbung. Online-Shops geben Kaufvorschläge zu Artikeln, die man gerade ansieht oder in den Warenkorb gelegt hat, um weitere Käufe anzuregen. Online-Video- oder Musik-Plattformen schlagen ähnliche Filme oder Musikstücke vor, um den Nutzer zu halten.\\
Die Daten, die nötig sind, um den User zu analysieren und zu bewerten, liefert dieser meistens selbst. Tracker verfolgen jeden Klick, jede Mausbewegung wird mit Timern analysiert, Warenkörbe werden ausgewertet, Cookies speichern Daten für die nächste Sitzung. Daraus lassen sich viele Informationen ableiten, man kann abschätzen, wen man zu Besuch hat und womit man Aufmerksamkeit und letztendlich Klicks erreichen kann. In vielen Bereichen liefern die Benutzer ganz bewusst Informationen. Wenn sie etwas bewerten, ihr Profil ausfüllen oder auf den "`like"'-Button drücken. Hierbei treten jedoch zwei Probleme auf. Sehr wenige Nutzer bewerten regelmäßig und extreme Bewertungen (sehr schlecht oder sehr gut) kommen häufiger vor, als die differenzierten Zwischenstufen.\\
Um verschiedene Musikstücke genauer bewerten zu können, setzt der Musik-Streaming-Dienst PANDORA\footnote{\url{www.pandora.com}} deshalb auf ein Expertensystem. Im Music Genome Project\footnote{\url{http://www.pandora.com/about/mgp}} werden 450 charakteristische Merkmale in Liedern von trainierten Musikern bewertet und analysiert. Studierte Musiker geben hier qualitativ hochwertige Bewertungen ab, um die Musikstücke korrekt zu kategorisieren.\\
Beim kollaborativen Filtern nutzt man diese Informationen über die User und Items, um Interessen zu vergleichen, Ähnlichkeiten zu bewerten und Vorschläge zu erstellen.\\ In dieser Arbeit stelle ich sechs Algorithmen vor und teste diese am MovieLens Datensatz 100k (s.\autoref{s.Datenmenge} \nameref{s.Datenmenge}), um sie miteinander vergleichen zu können.

\clearpage

