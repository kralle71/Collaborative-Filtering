\section{Vorhersage zu der Filmdatenbank MovieLens}\label{s.Test} \raggedbottom
Um die sechs Algorithmen miteinander vergleichen zu können, treten sie in verschiedenen Szenarien (s.  \autoref{s.split} \nameref{s.split}) gegeneinander an. Als Datenbank dient das MovieLens Dataset 100k. Als zweiten Test habe ich eine niedrig dimensionierte, randomisierte Matrix erstellt. Die Frage ist, ob man parallelen zwischen den beiden Datenmengen erkennt und dieses Wissen nutzen kann.

\subsection{Filmdatenbank Movielens}\label{s.Datenmenge}
Der verwendete Datensatz, auf dem die Algorithmen laufen, ist der MovieLens Datensatz 100k\footnote{\url{http://grouplens.org/datasets/movielens/}}. Dieser ist frei verfügbar und ist seit 1998 unverändert online, um erzielte Ergebnisse miteinander vergleichen zu können. Die Datensätze sind durch Userbewertungen, die auf Movielens\footnote{\url{https://movielens.org/}} abgegeben wurden, entstanden.
Im 100k Datensatz sind folgende Daten enthalten:
\begin{itemize}
	\item 943 User
	\item 1682 Filme
	\item 100.000 Ratings
	\item Ratings von 1-5
	\item Jeder User hat mindestens 20 Filme bewertet
	\item Demographische Informationen wie Geschlecht, Alter, Wohnort
\end{itemize}
Im Laufe der Jahre sind weitere, größere Datensätze entstanden. 
Ebenfalls verfügbar sind die Datensätze 1M mit einer Million Bewertungen von 6000 User zu 4000 Filmen. Im Jahr 2009 ist der 10M Datensatz zur Verfügung gestellt worden. Zusätzlich zu den 72.000 User und 10.000 Filmen wurden noch 100.000 Tags der Datenbank hinzugefügt. Diese Informationen helfen die Vorschläge stetig zu verbessern und zu optimieren. Im Jahr 2015 hat sich die Datenbank noch einmal verdoppelt auf 20M (27k Filme / 138k User / 100k Tags). Weiterhin gibt es zwei Datensätze die sich regelmäßig verändern und somit ständig neue Voraussetzungen bieten.


\subsection{Aufteilung der Datenmenge in Test- und Trainingsdaten}\label{s.split}
Die Datenmenge wurde aufgeteilt in Test- und Trainingsdaten. In den Trainingsdaten sind die Informationen enthalten, die der Algorithmus nutzen kann um Vorhersagen zu generieren. Die Testdaten werden genutzt um die Vorhersagen zu überprüfen mittels der echten Bewertungen der User. Jeder User hat mindestens 20 Filme bewertet. Um verschiedene Szenarien testen zu können werden $l$ Bewertungen pro User aus der Datenmenge in die Testmenge geschoben, mit $l \in \{1,5,10,19\}$. Alle anderen Bewertungen befinden sich in den Trainingsdaten.\\
Für das Test-Szenario $l = 1$ haben alle User noch mindestens 19 Bewertungen in den Trainingsdaten. Das heißt, dass sehr viele Informationen zur Verfügung stehen. Dies beeinflusst die Wahl ähnlicher User und natürlich das Wissen über einen bestimmten User. Wie wertet er im Mittel? Welche Genres findet er gut oder schlecht? Im Fall $l = 19$ haben einige Nutzer nur noch einen Film in der Trainingsmenge. In diesem Szenario wird es sehr viel schwieriger eine passende Vorhersage zu ermitteln.
	
\subsection{Matrix mit niedrigem Rang}
Als zweiter Testdatensatz dient eine zufällig erstellte Matrix mit niedrigem Rang. Die Frage, die sich hier stellt ist, kann man die Datenmenge auf eine Basis von typischen User reduzieren. Wenn man solche User finden kann, kann man mit weiteren Verfahren eine geeignete Linearkombination für jeden User finden, um approximativ die fehlenden Einträge der Matrix zu berechnen. In dieser Arbeit teste ich die oben genannten Algorithmen an einer solchen Menge, um eventuell Parallelen zwischen den beiden Datensätzen zu erkennen. Die Testmenge besteht aus 30 typischen, linear unabhängigen Usern die alle 1682 Filme zwischen 1 und 5 bewertet haben. Diese Matrix wird per Linearkombination der 30 Basis-User auf 943 User erweitert, mit einem Rang 30.
\begin{algorithm}[H]
o = 30 \\
m = 943 \\
n = 1682 \\
A1 = random.uniform(1,5,(m,o))\\
A2 = random.uniform(1,5,(o,n))\\
B = A1$^{T}$A2
\end{algorithm}
Matrix $B$ hat jetzt eine Größe von $943x1682$ und Rang 30. Um die Matrix an die MovieLens-Daten anzupassen, habe ich die Einträge in den Bereich 1 bis 5 transformiert und zufällig 93,7\% der Bewertungen entfernt, um die gleiche Dichte wie in den MovieLens-Daten (6,3\%) zu erreichen. Mit diesem Datensatz wird nur das Szenario $l = 1$ untersucht. Da der ganze Vorgang zufällig ausgewählt wird, ist in diesem Szenario nicht garantiert, dass über jeden User noch genug Informationen durch Bewertungen in der Trainingsmenge vorliegen.

\subsection{Berechnung des Fehlers}\label{s.error}
Die Güte der Algorithmen wird über den durchschnittlichen absoluten Fehler (Mean Absolute Error) ermittelt. Jede User-Item-Bewertung aus der Testmenge $R^{test}$ wird der Vorhersage des Algorithmus gegenüber gestellt. Der absolute Fehler wird aufsummiert und durch die Anzahl der Datensätze in der Testmenge dividiert.
\begin{equation}
		\mathrm{MAE} := \dfrac{\sum\limits_{(u,i)\in R^{test}}|r(u,i)-r^{test}_{u,i}|}{\mathrm{card}(R^{test})}  	\label{MAE}
\end{equation}
Ein MAE von 1 bedeutet demnach, dass der Algorithmus mit seinem Vorschlag um durchschnittlich einen Stern daneben liegt. Bei einem MAE von 0 ist die Testmenge perfekt getroffen. Die Bewertungen in der Datenmenge liegen im Ganzzahlbereich ${1,2,3,4,5}$. Die Algorithmen erstellen Ratings im Bereich $[1,5]\subset \mathbb{R}$. Dies führt zu einem Wertebereich des MAE von $[0,4]\subset \mathbb{R}$.

\clearpage